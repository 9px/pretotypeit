بالاخره تمام قطعات را جمع آوری کردیم، پس حالا میتوانیم چند مثال از ساختن
و تست پیش‌نمونه‌ها را بررسی کرده و براساس آنها تصمیم بگیریم. در هنگامی
خواندن مثال از اینکه راه‌های دیگری برای پیش‌نمونه‌سازی این ایده‌ها و تست
آنها به ذهنتان میرسد متعجب نشوید، زیرا یک راه برتر برای این‌کار وجود
ندارد. اگر راه‌های دیگری برای پیش‌نمونه سازی به ذهنتان نرسد برای من جای
تعجب دارد.

\section{مثال ۱: یک مشاهده گر سنجاب
کارکشته}\label{ux645ux62bux627ux644-ux6ccux6a9-ux645ux634ux627ux647ux62fux647-ux6afux631-ux633ux646ux62cux627ux628-ux6a9ux627ux631ux6a9ux634ux62aux647}

بیایید مثال خود را با پیش‌نمونه در جعلی بسازیم. همانطور که ممکن است به
یاد بیاورید، سندی به فکر نوشتن کتابی در مورد مشاهده سنجاب‌ها بود. از
آنجایی که نوشتن کتاب \emph{یک مشاهده گر سنجاب کار کشته} ماه‌ها زمان را
به خود اختصاص خواهد داد و او را از مشاهده‌ی سنجاب‌ها باز خواهد داشت. این
ایده خوبی است که کتاب را پیش‌نمونه سازی کند.

در مورد سندی، موفقیت کتاب تنها وابسته به تعداد افرادی است که کتاب را
میخرند(و به خرید مجدد آنها وابسته نیست) پس پیش‌نمونه‌سازی به منظور به
دست آوردن میزان علاقه اولیه کافی خواهد بود. پیش‌نمونه در جعلی برای این
حالت ایده‌آل خواهد بود. سندی اینگونه می‌تواند این کار را انجام دهد:

با ۱۰ دلار اون می‌تواند دامنه این کتاب(thecompeletesquirrelwatcher.com)
را بخرد و یک صفحه اولیه حاوی محتوای زیر بسازد:

\begin{quote}
علاقه مندان به سنجاب‌های عزیز
\end{quote}

\begin{quote}
از شما به خاطر علاقه‌ی‌تان به \emph{یک مشاهده گر سنجاب کار کشته} متشکرم.
من در حال کار روی این کتاب هستم، اما کتاب هنوز برای انتشار آماده نیست.
\end{quote}

\begin{quote}
برای رزور کردن یک نسخه از این کتاب با نرخ ویژه ۹/۹۸ دلار ایمیلی به:
iwantthebook@thecompeletesquirrelwatcher.com
\end{quote}

\begin{quote}
هنگامی که کتاب آماده شد در اولین فرصت به شما خبر خواهم داد.
\end{quote}

\begin{quote}
قیمت کتاب ۹/۹۸ دلار خواهد بود.
\end{quote}

\begin{quote}
در این زمان، اوقات خوشی در مشاهده سنجاب‌ها داشته باشید و حواستان به
واکسن هاری باشد!
\end{quote}

\begin{quote}
سندی(دختر سنجاب) واتسون
\end{quote}

همچنین او تبلیغی تحت وبی به شکل زیر ایجاد میکند

\begin{quote}
\textbf{آیا شما به مشاهده سنجاب‌ها علاقه دارید؟}
\end{quote}

\begin{quote}
thecompeletesquirrelwatcher.com
\end{quote}

\begin{quote}
کتابی برای مشاهده‌گران سنجاب حرفه‌ای
\end{quote}

\begin{quote}
نوشته شده توسط سندی واتسون. تنها ۹/۹۸ دلار
\end{quote}

با خرج کردن چند دلار، او می‌تواند تبلغ خود را در سایت‌هایی که به
سنجاب‌ها اختصاص یافته نشان دهد یا برای جستجوهای کلمات مرتبط با سنجاب به
نمایش گذاشته شود. وقتی افراد روی تبلغ او کلیک می‌کنند بصورت اتوماتیک به
سایت او انتقال می‌یابد.

این پیش‌نمونه در جعلی کمتر از ۵۰ دلار هزینه داشته و نیاز به چند ساعت کار
دارد. این کار نیازی به تخصص خاصی ندارد.

هنگامی که این پیش‌نمونه ایجاد شد، سندی می‌تواند یک ماه یا بیشتر صبر کند.
بعد از این زمان او می‌تواند داده‌هایی که از سرویس تبلیغات آنلاین بدست
آورده است را تحلیل کند. \textgreater{} تعداد افرادی که تبلیغ را
دیده‌اند: ۲۳۴۰۲ نفر

\begin{quote}
تعداد افرادی که روی تبلیغ کلیک کرده‌اند: ۶۳۴ نفر
\end{quote}

\begin{quote}
افرادی که ایمیل زده‌اند و گفته اند که کتاب را می‌خرند: ۲۳۰ نفر
\end{quote}

در اینجا چندین معیار میزان علاقه اولیه جالب قابل محاسبه است.

اولین معیار نشان‌دهنده این است که چند نفر سنجاب دوست به اندازه کافی
علاقه‌مند هستند تا روی تبلیغات مربوط به کتاب در مورد مشاهده سنجاب کلیک
کنند. اولین معیار میزان علاقه اولیه بصورت زیر محاسبه می‌شود:

\begin{quote}
اولین میزان علاقه اولیه = تعداد کلیک‌های روی تبلیغ / تعداد نمایش‌های
تبلیغ
\end{quote}

در این حالت این مقدار برابر ۶۳۴/۲۳۴۰۲ تقریبا ۲/۷ درصد است.

دومین معیار میزان علاقه اولیه درصد افرادی است که بعد از کلیک کردن رو
تبلیغ به اندازه کافی علاقه‌مند هستند که به سندی ایمیل میز‌نند.
\textgreater{} دمین میزان علاقه اولیه = تعداد ایمیل‌ها / تعداد
مشاهده‌های صفحه اول سایت

در این حالت برایبر ۳۵ درصد(۲۳۰/۶۳۴) است.

این یک عدد بسیار امیدوار کننده است. ۳۶ درصد افرادی که سایت سندی را
مشاهده‌کرده اند گفته‌اند که یک نسخه از کتاب سندی را می‌خواهند. درست است
که همه آنها کتاب را نخواهند خرید اما این عدد بسیار خوب است.

حال نوبت به تصمیم دشوار اینکه با توجه به این داده‌ها آیا سندی به نوشتن
کتاب بپردازد یا نه؟

این به انتظار سندی از کتاب وابسته است. داده‌ها می‌گوید که این کتاب خیلی
بعید است که به لیست کتاب‌های پرفروش نیویورک تایمز وارد شود بخاطر اینکه
تعداد افراد علاقه مند به سنجاب‌ها چندان نیستند. او به دنبال متخصص شدن و
مرجع شدن در این حوزه و فروش چند صد نسخه کتاب در سال است تا مخارج سفرهای
مشاهده‌ی سنجاب او تامین شود. در این حالت اطلاعات به او می‌گویند که که
\emph{یک مشاهده گر سنجاب کار کشته} احتمالا یک \textbf{چیز} \emph{درست}
برای تعداد کافی از آدم‌هاست تا سندی را خوشحال کند.

\section{مثال دوم: نرم افزار باب با اسم \emph{رتبه
بشقاب}}\label{ux645ux62bux627ux644-ux62fux648ux645-ux646ux631ux645-ux627ux641ux632ux627ux631-ux628ux627ux628-ux628ux627-ux627ux633ux645-ux631ux62aux628ux647-ux628ux634ux642ux627ux628}

برای مثال، فرض کنید که باب متخصص تغذیه است که می‌خواهد نرم‌افزار موبایلی
بنویسید که با تحلیل عکس یک وعده‌ی غذایی تحلیل میزان ارزش آن وعده را به
همراه یک امتیاز به کاربران بر می‌گرداند. امتیاز به عنوان مثال می‌تواند
«الف: سالم و ارزشمند» تا «و: هله، هوله». بگذارید این \textbf{چیز} باب را
\emph{رتبه بشقاب} بنامیم.

باب در مورد این نرم‌افزار با دوستان و افراد دیگر صحبت می‌کند، و بیشتر
آنها به او می‌گویند که این ایده‌ی عالی است و آنها قطعا از آن استفاده
خواهند کرد. خوشبختانه باب در مورد سرزمین فکر شنیده و می‌داند که نظرات
چقدر می‌تواند گمراه‌کننده باشد. او به قطع نمی‌داند که چه افرادی از این
نرم‌افزار استفاده کرده و حاضرند برای آن هزینه کنند. آیا کاربران به یاد
خواهند داشت که چند لحظه تامل کرده و عکسی از غذای خود قبل از خوردن آن
بگیرند؟ آیا آنها برای مدت محدودی به عنوان سرگرمی از آن استفاده خواهند
کرد و سپس آنرا فراموش خواهند کرد؟

باب همچنین می‌داند که توسعه یک سیستم نرم‌افزاری که واقعا بصورت اتوماتیک
یک وعده‌ی غذایی را براساس عکس آن تحلیل کند قطعا کار و هزینه بسیاری خواهد
برد و همچنین می‌داند رسیدن به نقطه‌ای که این نرم‌افزار به اندازه کافی
خوب و دقیق کار کند ممکن است غیر ممکن باشد(همانند مساله تبدیل گفتار به
متن آی بی ام)

مسائل باز بسیاری که بایستی پاسخی برای آنها یافت شود وجود دارد و تکنولوژی
آنها بسیار پر هزینه است. قطعا این \textbf{چیز} نیازمند پیش‌نمونه سازی
است.

\section{قدم اول: پیش‌نمونه‌های در جعلی و
پینوکیو}\label{ux642ux62fux645-ux627ux648ux644-ux67eux6ccux634ux646ux645ux648ux646ux647ux647ux627ux6cc-ux62fux631-ux62cux639ux644ux6cc-ux648-ux67eux6ccux646ux648ux6a9ux6ccux648}

تا الان شما نباید از اینکه من به عنوان اولین قدم در جعلی را پیشنهاد
داده‌ام متعجب باشید. باب بایستی به گونه‌ای در جعلی بسازد تا میزان علاقه
اولیه را اندازه گیری نماید(برای این منظور به مثال قبل مراجعه کنید).

بیایید فرض کنیم که داده‌های میزان علاقه اولیه امیدوار کننده است. اما،
چشم انداز و تعریف باب از موفقیت این نرم‌افزار علاوه بر علاقه اولیه
استفاده مداوم است(به عنوان مثال میزان علاقه مداوم ترغیب کننده). اگر
انجام آنچه نرم‌افزار نیازمند آن است سخت و عذاب آور باشد، افراد از انجام
آن سرباز خواهند زد. اصلا خود باب آنرا انجام خواهد داد؟ آیا باب به یاد
خواهد آورد که از غذایش قبل از شروع آن عکس بگیرد؟ آیا او از انجام اینکار
در حضور دیگران(خصوصا در رستوران‌ها) خجالت زده خواهد شد؟ آیا او تنها از
غذاهای سالم خود عکس خواهد گرفت و به راحتی دسر بستی موزی خود را فراموش
خواهد کرد؟

اگر خود ما به \textbf{چیز}مان ایمان نداشته و از آن استفاده نکنیم، چگونه
می‌توانیم دیگران را خالصانه راضی کرده یا انتظار داشته باشیم که آنها این
کار را انجام خواهند داد. برای پاسخ به این سوال، باب بایستی راهی را که جف
هاوکینز برای پیش‌نمونه‌سازی پالم پایلوت طی کرده است را دنبال کند. بایستی
یک یک نمونه پیش‌نمونه پینوکیو برای تست این ایده بصورت شخصی استفاده کند.
از آنجایی که باب تلفن هوشمندی دارای دوربین دارد او نیازی به رفتن به
کارگاه و ساختن یک بلوک چوبی ندارد. او به سادگی می‌تواند وانمود کند که
نرم‌افزار دوربین تلفن او همان نرم‌افزاری است که او علاقه‌مند به ساختن آن
است. او جاهای خالی را با تخیلات خود پر خواهد کرد.
