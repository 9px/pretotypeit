داستان ترجمه این کتاب از مسابقات موبایل امیرکیبر شروع شد که یکی از
میهمانان ویژه بصورت مختصر پیش‌نمونه‌سازی را معرفی کرده و از تجربه خود با
این روش گفت. این در کنار فضای پویای استارت‌آپ‌ها و راه‌اندازی کسب و کار،
باعث شد که جرقه ترجمه این متن به منظور افزایش محتوای فارسی در زمینه
کارآفرینی در ذهن من شکل گیرد. پس از مذاکرات اولیه با آقای ساویا و
استقبال ایشان از این پیشنهاد، ترجمه این کتاب در زمان‌های آزاد بنده آغاز
شد. پس از صرف مدت زمان قابل توجهی این ترجمه هم‌اکنون پیش‌روی شماست.
همانطور که خواندید این ترجمه نیز یک ترجمه کلاسیک نیست بلکه تنها یک
پیش‌نمونه است.

برای این ترجمه دوستان زیادی بنده را یاری رساندند. اولین و مهمترین
پیشتیبان من در این زمینه همسرم ،زهرا نوازی، بود که نقش قابل توجهی در
روانی و ویرایش این متن داشت. همچنین بهاره پرهیزکاری نیز زحمات زیادی در
ویرایش و اصلاح متن حاضر کشیده‌است. دوستان بسیاری نسخه‌های اولیه کتاب را
خوانده و مشوق بنده بوده‌اند. از همه آنها تشکر می‌کنم.
