شما الان یک ایده خام از آنچه پیش‌نمونه سازی درباره آن صحبت می‌کند دارید
و ما مثال‌های بیشتری را در فصل‌های آتی مطرح خواهیم کرد. اما قبل از این
مثال‌ها من قصد دارم اندکی زمان برای توضیح چرایی اهمیت زیاد پیش‌نمونه
سازی برای تمام ایده‌هایتان اختصاص می‌دهم.

آیا شما آمارهای ناامید کننده از بخش قبل را به یاد می‌آوردی

\begin{itemize}

\item
  ۹۰ درصد تمامی نرم‌افزارهای موبایل هیچ درآمدی ندارند.
\item
  هر چهار استارت‌آپ از پنج استارت آپ سرمایه سرمایه‌گذاران خود را از دست
  می‌دهند.
\item
  ۸۰ درصد رستوران‌های جدید در سال اول شکست می‌خورند.
\end{itemize}

اعداد دقیق ممکن است متفاوت باشد، اما پیغام این اعداد روشن است. با بیان
ساده اکثر \textbf{چیز}ها-که شام ایده‌ی شما می‌شود- تقدیرشان شکست است.
اکثر \textbf{چیزها} شکست می‌خورند بخاطر اینکه آنها \textbf{چیز}
\emph{غلط} هستند یعنی ایده‌هایی که ابتدا بصورت تئوری جالب به نظر
می‌رسیدند اما هنگامی که توسعه یافتند مشخص گردید که حتی آنها برخلاف آنچه
در ابتدا به نظر می‌رسد اندکی جالب، ترغیب کننده و یا کاربردی نبوده اند.

پیش‌نمونه سازی قدرت تبدیل یک \textbf{چیز} \emph{غلط} را به یک
\textbf{چیز} \emph{درست} ندارد و هیچ روش دیگری این امکان را نخواهد داشت.
اما پیش‌نمونه سازی امکان تشخیص \textbf{چیز}های \emph{غلط} بصورت سریع و
ارزان فراهم می‌کند پس شما می‌توانید \textbf{چیز}های جدید را امتحان
کنید(یا حتی نسخه‌های تغییر یافته \textbf{چیز}های فعلی) تا اینکه شما آن
\textbf{چیز} \emph{درست} گریزپا را بیابید.

از آنجایی که شکست دشمن ماست، و شناخت دشمن مهم است پس بیاید به شکست نگاه
دقیق‌تری داشته باشیم.

\section{قانون
شکست}\label{ux642ux627ux646ux648ux646-ux634ux6a9ux633ux62a}

شواهد در مورد وجود اتفاقات عجیب و غریب به ضرر \textbf{چیز}های جدید
اینقدر زیاد و قابل اعتماد است که می‌توانن قانونی برای آن اعلام کرد:

\textbf{قانون شکست}

اکثر \textbf{چیز}های جدید شکست می‌خورند، فارغ از اینکه چقدر بی نقص اجرا
شده باشند.

در این قانون کلمه «اکثر» اشاره به درصد بسیار زیاد ناامید کننده (معمولا
70-80-90 درصد) دارد و \textbf{چیز}ها تقریبا به هر چیزی که شما فکرش را
بکنید اطلاق می‌شود: استارت‌آپ‌ها، رستوران‌ها، فیلم‌ها، کتاب‌ها،
نوشابه‌ها، سریال تلویزیونی و غیره. و آن \textbf{چیز} شما در یکی از این
دسته‌های قرار گرفته و قطعا دچار همان بدبیاری‌هایی می‌شود که مابقی
\textbf{چیز}های دیگر دچار می‌شوند.

من هم اکنون شکایت‌های شما را مبنی برا اینکه «خب این قانون چگونه به ما
کمک خواهد کرد؟ این قانون به ما می‌گوید ما به احتمال زیاد شکست می‌خوریم
حتی اگر ما بسیار خوب روی \textbf{چیز}مان کار کرده باشیم. این قانون تنها
به ما بدبیاری می‌دهد و ما را پا در هوا نگه می‌دارد. تنها کاری که این
قانون انجام می‌دهد روحیه ما را پایین آورده و انیزه ما را می‌کشد.»

در ظاهر این حرف درست است و قانون شکست به نظر کمک کننده نمی‌رسد. وقتی
بصورت دقیق صحبت کنیم این قانون حتی یک قانون دقیق نیست. شما می‌توانید
نیوتن را هنگامی که در حال مشاهده جاذبه بود تصور کنید که می‌گوید:
«احتمالا بیشتر اجسامی که رها می‌شوند سقوط می‌کنند؟» اما به دست آوردن این
قانون به نسبت آسان است. او در حال بررسی و مشاهده یک قانون تغییر ناپذیر و
عمومی طبیعی بود. اما در طرف دیگر موفقیت بازار یک محصول مرتبط با رفتار
انسانی است که بسیار متغیر، بی ثبات و (و در اغلب موارد) غیر منطقی است. در
این موضوع، فرموله سازی احتمالی قانون شکست بهترین چیزی است که به دست
می‌آید.

من باور دارم با اینکه \emph{قانون شکست} فاصله زیادی تا کاملا دقیق بودن
دارد، اما از اهمیت زیادی هم برخوردار است. اگر شما این درستی این قانون را
قبول کرده و یا حتی در اکثر اوقات آنرا بپذیرید، و بدانید که خود شما یا آن
\textbf{چیز} شما از این قاعده مستثنی نیستند، نظر شما از «بیایید آن
\textbf{چیز} را انجام دهیم! بیایید آن \textbf{چیز} را بسازیم» به نظر
محتاطانه «بیایید آن \textbf{چیز} را بیازماییم. بیایید پیش‌نمونه آن
\textbf{چیز} را بسازیم» تغییر می‌کند.

من میدانم که «آنرا انجام دهیم» بسیار حذاب و قهرمانانه است. «با مساله
گلاویز شدن»، «شرط بندی زمین‌ها» و «بادبان‌ها رو بکشید» سرآغاز افسانه‌های
بسیاری بودند اما ابتدای شروع شکست‌های فاجعه انگیز نی هستند.

با توجه به آنچه گفته شد، ممکن است وقت‌هایی باشد که شما بخواهید به
اتفاقات بد بی‌توجه باشید و تنها بخواهید بدون در نظر گرفتن نتیجه روی آن
\textbf{چیز} خود فعالیت کنید. من به هیچ وجه شما را از این روش دلسرد
نمی‌کنم. لااقل به ندرت در زندگی‌مان، ما بایستی ریسک‌های بزرگ کرده و تنها
برای رسیدن به آن \textbf{چیز} به پیش برانیم. زمان‌هایی خواهد بود که شما
ساختن یک \textbf{چیز} خاص برای شما مهم‌تر از داشتن یک \textbf{چیز}
\emph{درست} است. ار شما در این وضعیت هستید، لبخندی به \emph{قانون شکست}
بزنید، احتیاط را کنار بگذارید و این کتاب را در زباله بیاندازید و با تمام
قلب روح خود روی آن \textbf{چیز} کار کنید. خدایارتان باشد! من طرف شما
هستم و آرزو میکنم موفق شوید.

اما اگر از سوی دیگر، شما در موقعیتی هستید که ۱۰۰ درصد به یک \textbf{چیز}
خاص پایبند نیستید، بیشینه کردن شانس موفقیت بسیار حساس است. به
\emph{قانون شکست} احترام لازم را بگذارید زیرا \ldots{}

\section{\ldots{} شکست یکی از انتخاب‌ها
نیست}\label{ux634ux6a9ux633ux62a-ux6ccux6a9ux6cc-ux627ux632-ux627ux646ux62aux62eux627ux628ux647ux627-ux646ux6ccux633ux62a}

این درست است. برای هر \textbf{چیز}، شکست یکی از انتخاب‌ها نیست بلکه
محتمل‌ترین خروجی است.

ما نمی‌توانیم از \emph{قانون شکست} فرار کنیم. ما نمی‌توانیم شانس
\textbf{چیز}های جدید را تغییر دهیم.

اما آنچه ما می‌توانیم انجام دهیم این است که از \emph{قانون شکست} به نفع
خودمان استفاده کنیم همانگونه که حسابداران از قوانین مالیاتی استفاده
می‌کنند و لیدی گاگا از پاپاراتزی‌ها.

چگونه می‌توانیم این را انجام دهیم؟

ما شکست را دعوت می‌کنیم، ما به دنبال شکست می‌رویم، ما آنرا شکار می‌کینم
تا چهره کریه خود را در اولین فرصت ممکن به ما نشان بدهد تا ما بدانیم که
راه غلطی را طی می‌کنیم. پس بتوانیم در زودترین موقع تغییرات لازم را انجام
دهیم.

ما چندین طعمه در شکل پیش‌نمونه جعل می‌کنیم. بعضی وقت‌ها آنها شبیه یا بوی
آن \textbf{چیز} ما را می‌دهد. چیزی که ما بتوانیم با کمک آن \emph{دیو
شکست} را مجبور به نشان دادن سر کریه‌ش بکنیم. ما به در غار تاریک و
نم‌ناکی که دیو در آن ساکن است می‌رویم. آنگاه طعمه پیش‌نمونه خود را در
ورودی این غار گذاشته تا ببینیم که دیو به طعمه نزدیک می‌شود که ما بتوانیم
بوی بد تنفسش را بشنویم و بتوانیم نگاه کوتاهی به به دهان بی‌رحمش و چشمان
کوچکش بیاندازیم. آنقدر نزدیک که مطمئن شویم که دیو واقعی است. آنگاه طعمه
ارزان قیمت خود را به عنوان قربانی به دیو تقدیم کرده و در جهت مخالف فرار
کنیم. قبل از اینکه دیو بتوانی دندان‌های تیزش را در گوشت ما فرو کرده و ما
را به درون غار خود کشیده و از ما جشنی برپا کند.
