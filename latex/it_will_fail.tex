شما الان یک ایده خام از آنچه پیش‌نمونه سازی درباره آن صحبت می‌کند دارید
و ما مثال‌های بیشتری را در فصل‌های آتی مطرح خواهیم کرد. اما قبل از این
مثال‌ها من قصد دارم اندکی زمان برای توضیح چرایی اهمیت زیاد پیش‌نمونه
سازی برای تمام ایده‌هایتان اختصاص می‌دهم.

آیا شما آمارهای ناامید کننده از بخش قبل را به یاد می‌آورید

\begin{itemize}

\item
  ۹۰ درصد تمامی نرم‌افزارهای موبایل هیچ درآمدی ندارند.
\item
  هر چهار استارت‌آپ از پنج استارت آپ سرمایه سرمایه‌گذاران خود را از دست
  می‌دهند.
\item
  ۸۰ درصد رستوران‌های جدید در سال اول شکست می‌خورند.
\end{itemize}

اعداد دقیق ممکن است متفاوت باشد، اما پیغام این اعداد روشن است. با بیان
ساده اکثر \textbf{چیز}ها-که شامل ایده‌ی شما می‌شود- تقدیرشان شکست است.
اکثر \textbf{چیزها} شکست می‌خورند بخاطر اینکه آنها \textbf{چیز}
\emph{غلط} هستند یعنی ایده‌هایی که ابتدا بصورت تئوری جالب به نظر
می‌رسیدند اما هنگامی که توسعه یافتند مشخص گردید که حتی آنها برخلاف آنچه
در ابتدا به نظر می‌رسد اندکی جالب، ترغیب کننده و یا کاربردی نبوده اند.

پیش‌نمونه سازی قدرت تبدیل یک \textbf{چیز} \emph{غلط} به یک \textbf{چیز}
\emph{درست} را ندارد و هیچ روش دیگری این امکان را نخواهد داشت. اما
پیش‌نمونه سازی امکان تشخیص \textbf{چیز}های \emph{غلط} را بصورت سریع و
ارزان فراهم می‌کند، پس شما می‌توانید \textbf{چیز}های جدید را امتحان
کنید(یا حتی نسخه‌های تغییر یافته \textbf{چیز}های فعلی) تا اینکه شما آن
\textbf{چیز} \emph{درست} گریزپا را بیابید.

از آنجایی که شکست دشمن ماست، و شناخت دشمن مهم است پس بیاید به شکست نگاه
دقیق‌تری داشته باشیم.

\section{قانون
شکست}\label{ux642ux627ux646ux648ux646-ux634ux6a9ux633ux62a}

شواهد در مورد وجود اتفاقات عجیب و غریب به ضرر \textbf{چیز}های جدید
اینقدر زیاد و قابل اعتماد است که می‌توانن قانونی برای آن اعلام کرد:

\textbf{قانون شکست}

اکثر \textbf{چیز}های جدید شکست می‌خورند، فارغ از اینکه چقدر بی نقص اجرا
شده باشند.

در این قانون کلمه «اکثر» اشاره به درصد بسیار زیاد ناامید کننده (معمولا
\lr{70-80-90} درصد) دارد و \textbf{چیز}ها تقریبا به هر چیزی که شما فکرش
را بکنید اطلاق می‌شود: استارت‌آپ‌ها، رستوران‌ها، فیلم‌ها، کتاب‌ها،
نوشابه‌ها، سریال تلویزیونی و غیره. و آن \textbf{چیز} شما در یکی از این
دسته‌های قرار گرفته و قطعا دچار همان بدبیاری‌هایی می‌شود که مابقی
\textbf{چیز}های دیگر دچار می‌شوند.

من هم اکنون شکایت‌های شما را مبنی برا اینکه «خب این قانون چگونه به ما
کمک خواهد کرد؟ این قانون به ما می‌گوید ما به احتمال زیاد شکست می‌خوریم
حتی اگر ما بسیار خوب روی \textbf{چیز}مان کار کرده باشیم. این قانون تنها
به ما بدبیاری می‌دهد و ما را پا در هوا نگه می‌دارد. تنها کاری که این
قانون انجام می‌دهد روحیه ما را پایین آورده و انگیزه ما را می‌کشد.»

در ظاهر این حرف درست است و قانون شکست به نظر کمک کننده نمی‌رسد. وقتی
بصورت دقیق صحبت کنیم این قانون حتی یک قانون دقیق نیست. شما می‌توانید
نیوتن را هنگامی که در حال مشاهده جاذبه بود تصور کنید که می‌گوید:
«احتمالا بیشتر اجسامی که رها می‌شوند سقوط می‌کنند؟» اما به دست آوردن این
قانون به نسبت آسان است. او در حال بررسی و مشاهده یک قانون تغییر ناپذیر و
عمومی طبیعی بود. اما در طرف دیگر موفقیت بازار یک محصول مرتبط با رفتار
انسانی است که بسیار متغیر، بی ثبات و (و در اغلب موارد) غیر منطقی است. در
این موضوع، فرموله سازی احتمالی قانون شکست بهترین چیزی است که به دست
می‌آید.

من باور دارم با اینکه \emph{قانون شکست} فاصله زیادی تا کاملا دقیق بودن
دارد، اما از اهمیت زیادی هم برخوردار است. اگر شما درستی این قانون را
قبول کرده و یا حتی در اکثر اوقات آنرا بپذیرید، و بدانید که خود شما یا آن
\textbf{چیز} شما از این قاعده مستثنی نیستند، نظر شما از «بیایید آن
\textbf{چیز} را انجام دهیم! بیایید آن \textbf{چیز} را بسازیم» به نظر
محتاطانه «بیایید آن \textbf{چیز} را بیازماییم. بیایید پیش‌نمونه آن
\textbf{چیز} را بسازیم» تغییر می‌کند.

من میدانم که «آنرا انجام دهیم» بسیار جذاب و قهرمانانه است. «با مساله
گلاویز شدن»، «شرط بندی زمین‌ها» و «بادبان‌ها رو بکشید» سرآغاز افسانه‌های
بسیاری بودند اما ابتدای شروع شکست‌های فاجعه انگیز نیز هستند.

با توجه به آنچه گفته شد، ممکن است وقت‌هایی باشد که شما بخواهید به
اتفاقات بد بی‌توجه باشید و تنها بخواهید بدون در نظر گرفتن نتیجه روی آن
\textbf{چیز} خود فعالیت کنید. من به هیچ وجه شما را از این روش دلسرد
نمی‌کنم. لااقل به ندرت در زندگی‌مان، ما بایستی ریسک‌های بزرگ کرده و تنها
برای رسیدن به آن \textbf{چیز} به پیش برانیم. زمان‌هایی خواهد بود که شما
ساختن یک \textbf{چیز} خاص برای شما مهم‌تر از داشتن یک \textbf{چیز}
\emph{درست} است. اگر شما در این وضعیت هستید، لبخندی به \emph{قانون شکست}
بزنید، احتیاط را کنار بگذارید و این کتاب را در زباله بیاندازید و با تمام
قلب و روح خود روی آن \textbf{چیز} کار کنید. خدایارتان باشد! من طرف شما
هستم و آرزو میکنم موفق شوید.

اما اگر از سوی دیگر، شما در موقعیتی هستید که ۱۰۰ درصد به یک \textbf{چیز}
خاص پایبند نیستید، بیشینه کردن شانس موفقیت بسیار حساس است. به
\emph{قانون شکست} احترام لازم را بگذارید زیرا \ldots{}

\section{\ldots{} شکست یکی از انتخاب‌ها
نیست}\label{ux634ux6a9ux633ux62a-ux6ccux6a9ux6cc-ux627ux632-ux627ux646ux62aux62eux627ux628ux647ux627-ux646ux6ccux633ux62a}

این درست است. برای هر \textbf{چیز}، شکست یکی از انتخاب‌ها نیست بلکه
محتمل‌ترین خروجی است.

ما نمی‌توانیم از \emph{قانون شکست} فرار کنیم. ما نمی‌توانیم شانس
\textbf{چیز}های جدید را تغییر دهیم.

اما آنچه ما می‌توانیم انجام دهیم این است که از \emph{قانون شکست} به نفع
خودمان استفاده کنیم همانگونه که حسابداران از قوانین مالیاتی استفاده
می‌کنند و لیدی گاگا از پاپاراتزی‌ها.

چگونه می‌توانیم این را انجام دهیم؟

ما شکست را دعوت می‌کنیم، ما به دنبال شکست می‌رویم، ما آنرا شکار می‌کینم
تا چهره کریه خود را در اولین فرصت ممکن به ما نشان بدهد تا ما بدانیم که
راه غلطی را طی می‌کنیم. پس بتوانیم در زودترین موقع تغییرات لازم را انجام
دهیم.

ما چندین طعمه در شکل پیش‌نمونه جعل می‌کنیم. بعضی وقت‌ها آنها شبیه یا بوی
آن \textbf{چیز} ما را می‌دهد. چیزی که ما بتوانیم با کمک آن \emph{دیو
شکست} را مجبور به نشان دادن سر کریه‌ش بکنیم. ما به در غار تاریک و نمناکی
که دیو در آن ساکن است می‌رویم. آنگاه طعمه پیش‌نمونه خود را در ورودی این
غار گذاشته تا ببینیم که دیو به طعمه نزدیک می‌شود که ما بتوانیم بوی بد
تنفسش را بشنویم و بتوانیم نگاه کوتاهی به به دهان بی‌رحمش و چشمان کوچکش
بیاندازیم. آنقدر نزدیک که مطمئن شویم که دیو واقعی است. آنگاه طعمه ارزان
قیمت خود را به عنوان قربانی به دیو تقدیم کرده و در جهت مخالف فرار کنیم.
قبل از اینکه دیو بتواند دندان‌های تیزش را در گوشت ما فرو کرده و ما را به
درون غار خود کشیده و از ما جشنی برپا کند.

بهترین کاری که شما می‌توانید انجام دهید غذا دادن به این دیو توسط
لقمه‌های کوچک و ارزان از \textbf{چیز}های گوناگون است. این دیو علاقه‌مند
به خوردن \textbf{چیز}های غلط است اما در صورت فرصت یافتن مشتاق خوردن
شماست! شما بایستی آمادگی انداختن لقمه‌های ساخته شده از \textbf{چیز}ها و
فرار، را داشته باشید. اگر شما این آمادگی را نداشته باشید، اگر به آن
\textbf{چیز}تان وابسته شوید، احتمالا در نهایت دیو تمام زمان و تلاش شما
را خواهد بلعید.

اگر ما اینکار را به درستی انجام دهیم، تنهای چیزی که از دست می‌دهیم تنها
طعمه (پیش‌نمونه) ماست، و یک روز دیگر وقت داریم تا یک \textbf{چیز} دیگر
را امتحان کنیم تا زمانی که \textbf{چیز}ی بیابیم که دیو را به خود جذب
نکند. طعمه‌ای که ممکن است مبدل به یک \textbf{چیز} \emph{درست} شود.

دنبال کردن ایده‌تان تا سرانجام، حتی اگر پایان خوبی نداشته باشد و به این
نتیجه برسید که ایده‌ی غلطی بوده‌است، ممکن است هیجان انگیز و قهرمانانه
باشد. اما پیش‌نمونه سازی حداقل به همین اندازه هیجان‌انگیز است. در
پیش‌نمونه سازی، شما هنوز در حالا انجام جستجوی حماسی و پرچالش هستید،
جستجویی برای یافتن یک \textbf{چیز} \emph{درست}. میان شما و \textbf{چیز}
\emph{درست} دیو ترسناک شکست ایستاده است. شما نمی‌توانید از این دیو دوری
کنید. اما شما باید با آن بجنگید - اما با کمک پیش‌نمونه‌سازی شانس پیروزی
شما بسیار بیشتر است.

این ذات استراتژی ماست - ذات اصلی پیش‌نمونه سازی. اما بازی کردن با شکست
تنها در حالتی منطقی است که از طعمه‌های آسان و ارزان استفاده کنیم.
پیش‌نمونه‌هایی که با حداقل هزینه در چند ساعت یا روز درست شده‌اند و رها
کردن آنها برای ما مهم نیست.

\section{سه راه برای
شکست}\label{ux633ux647-ux631ux627ux647-ux628ux631ux627ux6cc-ux634ux6a9ux633ux62a}

شکست محتمل‌ترین نتیجه هر یک از \textbf{چیز}های شماست، اما تمام شکست‌ها
یکی نیستند. سه راه برای پیش‌بردن \textbf{چیز}تان دارید. سه راه برای کنار
آمدن با دیو شکست:

\begin{itemize}

\item
  هیچ کاری در مورد آن \textbf{چیز} انجام ندهید
\item
  آن \textbf{چیز} را انجام دهید(نمونه محصول سازی)
\item
  آن \textbf{چیز} را امتحان کنید.
\end{itemize}

اولین راه روش مورد استفاده تنبلان و بزدلان است: افراد یا شرکت‌هایی که
تنبل‌تر، سست‌تر، یا بزدل تر از آن هستند که تلاشی در مورد هر چیزی انجام
دهند. کنار آمدن با شکست با تلاش نکردن مطمئن ترین روش برای همیشه شکست
خوردن است. اگر شما تا اینجای کتاب را خوانده‌اید قطعا جز این دسته نیستید.
شما آماده ساختن چیزی هستید.

دوم راه شکست خوردن دقیقا عکس روش ااول است. برخلاف تنبلی، سستی یا بزدلی
شما تلاش، اطمینان و گستاخی بیش از حدی دارید. مواجهه با شکست از طریق دسته
کم گرفتن آن، در اکثر موارد منجر به شکست کُند، پرهزینه و دردناک خواهد شد.

این دو نوع شکست معمولا به دلیل فکر کردن بیش از حد، حرف زدن بیش از حد و
توجه کم و خیلی دیر به واقعیت است. همه \textbf{چیز}ها در قالب یک ایده به
دنیا می‌آیند، اما اگر ما به سرعت از فکر کردن و حرف زدن تغییر رویه ندهیم
آن \textbf{چیز}ما در جای بسیار خطرناکی قرار گرفته است. این تغییر رویه
بدین صورت است که شما باید هر چه سریع‌تر یک مورد عینی در مقابل کاربران و
مشتریان بالقوه خود قرار دهید. این جای خطرناک را من \emph{سرزمین فکر}
می‌نامنم.

\section{سرزمین
فکر}\label{ux633ux631ux632ux645ux6ccux646-ux641ux6a9ux631}

سرزمین فکر یک سرزمین ساختگی است که دو نوع موجود غریب در آن ساکن هستند و
بایکدیگر در تعامل هستند: ایده‌ها و نظرات. بصورت دقیق‌تر: ایده‌های
\emph{تحقق نیافته} و نظرات مربوط به این ایده‌ها.

سرزمین فکر جایی است که همه \textbf{چیز}ها به عنوان یک ایده‌ی ساده، خالص
و انتزاعی شروع می‌شود. وقتی این ایده‌ها در این محیط شناور هستند نظرات را
به خود جذب می‌کنند همانند بارنکل‌ها(نوعی صدف) که به کشتی می‌چسبند.

سرزمین فکر جای بسیار امنی برای ایده‌هاست، زیرا آنها تا تبدیل به فرم
محسوسی همانند یک نمونه اولیه خام یک نرم‌افزار یا نسخه اول کتاب نشوند،
نمی‌توانند شکست بخورند. تنها چیزی که یک ایده انتزاعی می‌تواند «تولید
کند» نظرات است. نظرات حتی بیشتر از ایده‌ها انتزاعی و دوپهلو هستند.

برخلاف امنیتی که سرزمین فکر برای ایده‌ها ایجاد می‌کند، جای بسیار خطرناکی
برای سازندگان، مبتکران، کارآفرینان و نویسندگان است. نظراتی که در سرزمین
فکر جمع شده و به ایده‌های ما می‌چسبند می‌توانند از دو راه منجر به شکست
می‌شوند:

نظرات \emph{غلط منفی} در مورد \textbf{چیز}هایمان ممکن است باعث ترس ما
شده و باعث بشوند ما در مورد \textbf{چیز}مان کاری انجام ندهیم.

نظرات \emph{غلط مثبت} در مورد \textbf{چیز}هایمان ممکن باعث نادیده گرفتن
\emph{قانون شکست} شده و بیش از حد و زود متعهد به ایده‌ی مان شویم.

بیایید به این دو سناریو را زودتر بررسی کنیم.

\section{سناریو «کاری انجام
ندادن»}\label{ux633ux646ux627ux631ux6ccux648-ux6a9ux627ux631ux6cc-ux627ux646ux62cux627ux645-ux646ux62fux627ux62fux646}

بیشتر \textbf{چیز}ها هیچگاه از سرزمین فکر خارج نمی‌شوند. آنها برای همیشه
در این برزخ به عنوان ایده‌های تحقق نیاافته می‌مانند. این ناراحت کننده
ترین شکل شکست است. قطعا احتمال اینکه این \textbf{چیز}ها غلط باشند زیاد
است، اما احتمال کوچکی هم وجود دارد که پالم پایلوت بعدی، گوگل و یا توییتر
بعدی بوده، و کسی آنرا بدون امتحان کردن رها کنند. بسیار، بسیار، بسیار
ناراحت کننده است.

درصدی خوبی از \textbf{چیز}ها از دیدن خورشید روز باز می‌مانند بخاطر اینکه
\emph{ایده‌پردازانشان} از سرجای خود برای انجام کار روی آنها بلند
نمی‌شوند. آنها بر این باورند که ایده‌ی آنها برنده است. دیگران نیز به
آنها می‌گویند که ایده‌ی آنها برنده است اما آنها بسیار تنبل/ خسته/ مشغول/
ورشکسته/ بی‌تجربه/ ترسان/ (عذر دلخواهتان را در اینجا قرار دهید) تر از آن
هستند که برای \textbf{چیز}هایشان کاری انجام دهند. همانطور که در ادامه
خواهیم دید، پیش‌نمونه‌سازی به ما کمک خواهد کرد که با این وضعیت‌ها مواجه
شویم.

درصد باقی‌مانده \textbf{چیز}ها از دیدن خورشید روز باز می‌مانند بخاطر
اینکه ما تنبل/خسته/مشغول/ \ldots{} نیستیم اما بخاطر این است که هنگامی که
در سرزمین فکر هستند، \textbf{چیز}هایمان نظرات منفی کافی(کمی نظرات منفی
خودمان و بیشتر نظرات منفی دیگران) به خود جذب کرده تا منجر به متزلزل شدن
نظر ما نسبت به آن \textbf{چیز} و فروریختگی آن شوند. این وضعیت بسیار
اتفاق می‌افتد و متاسفانه گریبان بسیاری از \textbf{چیز}های \emph{درست} را
می‌گیرد. چگونه این اتفاق می‌افتد؟ بگذارید مثالی بزنم:

بیایید فرض کنیم که آلیس ایده‌ای برای یک نرم‌افزار موبایل جدید دارد، چیزی
که با استفاده از پیغام‌های متنی به افراد اجازه می‌دهد پیغام‌های
کوتاهی(حداکثر ۱۰۰ تا ۲۰۰ کاراکتر) ارسال نموده که بصورت اتوماتیک به دست
افراد فامیل یا هرکسی که میخواهد دنبال کننده ما باشد میرسد. بگذارید نام
این نرم‌افزار را \emph{ربات متنی} به نامیم.

آلیس ایده \emph{ربات متنی} خود(\textbf{چیز} خود) را به سرزمین فکر‌
میبرد. بگذارید ببینم چه اتفاقی می‌افتد:

‍\emph{آلیس ایده ربات متنی خود را با مجموعه‌ای از دوستان در میان گذاشته
و نظرات آنها را جویا می‌شود}

\emph{تقریبا تمام دوستان او می‌گویند که این یک ایده بی‌مزه بوده و آنها
هیچگاه از آن استفاده نخواهند کرد:}

\emph{«چه کسی به کاری که تو در حال انجام آن هستی اهمیت می‌دهد؟»}

\emph{«چرا من باید تو را دنبال کنم؟»}

\emph{«من دوست ندارم که دنبال شوم.»}

\emph{«چرا بایستی متن به ۱۰۰ یا ۲۰۰ کاراکتر محدود شود. این احمقانه
است.»}

\emph{دوستانی که نمیخواهند بصورت کامل منفی‌باف باشند پیشنهاداتی در
راستای بهبود می‌دهند: «شاید بهتر است بیخیال محدودیت ابلهانه تعداد
کاراکترها شده و قبل انتشار آن اجازه بدهی که عکس و مختصات خود را نیز به
اشتراک بگذارند»}

\emph{دوستان بی مبالات. آنها در این مورد چه چیزی می‌دانند؟ آلیس تصمیم
می‌گیرد که} \textbf{چیز} \emph{خود را به سرمایه گزاران پرخطر که کارشان
این است ببرد. آنها خواهند دید که ایده‌اش چقدر خوب است.}

\emph{سرمایه گذاران پرخطر نیز} \textbf{چیز} \emph{او را در نمی‌یابند.
برخی تنها می‌گویند: «این برای ما به اندازه کافی بزرگ نیست، اما موفق
باشی». برخی در مورد اطلاعات کاربران می‌پرسند، اما آلیس چیزی در اختیار
ندارد: «ببخشید، در حال حاضر این تنها یک ایده‌ است، اما به اسلایدهای من
نظری بیاندازید \ldots{}» سرمایه گزاران به آلیس می‌گویند که «\ldots{}
وقتی که یک میلیون یوزر داشتی برگرد و ما آنگاه صحبت خواهیم کرد.»}

وای. آلیس چگونه می‌تواند فکر کند که این ایده حتی خوب است. کار خوبی کرده
است که پیش از استعفا از کارش و توسعه این نرم‌افزار بدرد نخور نظرات
دیگران را پرسیده است. او تصمیم میگیرد که آنرا فراموش کند. خدا را شکر!
نزدیک بود۱

این وضعیت بسیار اتفاق می‌افتد! البته از آنجا که \textbf{چیز}های غلط زیاد
هستند، این نظرات منفی بسیاری از ایده‌های غلط را می‌کشند. اما آنها بسیاری
از \textbf{چیز}های بی‌گناه و بسیار امیدوار کننده را نیز می‌کشند.

بیشتر شما احتمالا به این نتیجه رسیده‌اید که مثال برنامه \emph{ربات متن}
آلیس یک توصیف اندکی غیر مستقیم به توییتر بود. توییتر قطعا یکی از
موفق‌ترین محصولات با تغییرات اساسی در دنیا در طول تاریخ بوده است.

اما با این حال، قبل از اثبات کارایی و تاثیر توییتر واضح و غیر قابل انکار
شود، نظرات اولیه و عکس العمل بسیاری از مردم -حتی بسیاری از سرمایه‌گذاران
پرخطر و سرمایه گزاران باهوش- نسبت به این ایده منفی بود: آنها آنرا درک
نمی‌کردند. الان هم بسیاری از مردم هستند که آنرا درک نمی‌کنند، اما این
موضوع مهم نیست بخاطر اینکه ده‌ها میلیون نفر آنرا درک کرده و از آن
استفاده می‌کنند. توییتر یک \textbf{چیز} \emph{درست} بود اما اینرا نمی‌شد
از پذیرشش در سرزمین فکر دریافت.

نظرات ، اه!

\section{سناریو «انجامش
بده»}\label{ux633ux646ux627ux631ux6ccux648-ux627ux646ux62cux627ux645ux634-ux628ux62fux647}

ما دیدم که نظرات منفی ممکن است منجر به کشته شدن بسیاری از
\textbf{چیز}های \emph{درست} در سرزمین افکار شود. اما این تنها نصف داستان
است. بیایید به آنروی سکه نگاهی بیاندازیم و ببینیم چگونه نظرات مثبت منجر
به تعهد زیاد برروی \textbf{چیز}های \emph{غلط} شود.

ما نیاز به \textbf{چیز} مثالی دیگری داریم.

نظراتان در مورد این مثال چیست: تام، یک نرم‌افزار نویس درجه یک، ایده‌ای
برای یک نرم‌افزار موبایل دارد که در آن به افرادی با مشکلات رمانتیک
همانند خودش کمک کند. آین نرم افزار بصورت اتوماتیک در زمان‌های تصادفی
پیغام‌های معنی‌داری به دیگری مهم ارسال می‌نماید. بیایید این نرم‌افزار را
\emph{ربات متن قشنگ} بنامیم. دیگری مهم شما پیغام‌هایی همانند این
پیغام‌ها دریافت می‌کنند: «سلام عزیزم. من به تو فکر می‌کنم. عشق. همستر
کوچک تو.» یا «عزیزکم، من به تو پیغام دادم تا فقط بگویم دوستت دارم. بوس
بوس بوس»

برنامه \emph{ربات متن قشنگ} تام باعث می‌شود که دیگران مهم فکر کند که
فردی با مشکلات رمانتیک در حال حاضر به آنها فکر می‌کند - در حالی که ممکن
است آنها با دوستان خود بیرون رفته باشند یا در حال مشاهده کشتی باشند.
چقدر رمانتیک!

این \textbf{چیز} تام است، ایده جدیدی که روی آن فکر می‌کنند.

این اتفاقی است که برای ایده‌ی تام در سرزمین فکر می‌افتد:

\emph{تام ایده‌ی خود را در مورد ربات متن قشنگ به دوستان و همکاران خود(که
همه مرد هستند) می‌گوید و نظر آنها را جویا می‌شود. او به این کار «تحقیقات
بازار» اتلاق می‌کند.}

\emph{بیشتر دوستان تام بگذارید بگوییم ۷۰ درصد انها فکر می‌کنند که این یک
ایده خفن است و به تام می‌گویند که آنها قطعا آنرا به ارزش ۱.۹۹ دلار
می‌خرند و همچنین بصورت مداوم از آن استفاده می‌کنند.}

\emph{تام از «تحقیقات بازار» به کمک استقرا به این نتیجه میرسد که او به
راحتی می‌تواند با کمک نرم‌افزار خود میلیونر شود. زیرا ۷۰ درصد مردان
ضربدر ۱.۹۹ دلار عدد بسیار بزرگی خواهد بود.}

\emph{به پشتوانه نظرات متخصصان این حوزه و تحلیل مالی، تام از شغلش استعفا
داده، سه ماه زمان و تمام پس‌انداز خود را صرف نوشتن نسخه‌ی با تمام
امکانات و بسیار شیک از نرم افزار ربات متن قشنگ می‌کند. تام یک توسعه
دهنده بزرگ است و سررشته خوبی در طراحی دارد پس نرم‌افزار زیبا بوده و بدون
نقص کار می‌کند. اولین نسخه متن‌های عاشقانه کوتاهی را به بیش از ۲۰ زبان
دنیا ارسال می‌کند! برای پوشش تمام حوزه‌ها و پیش‌دستی در رقابت او
نرم‌افزار را برای همه پلتفورم‌های موبایل(اندروید، آیفون و بلک‌بری) بصورت
همزمان ارائه می‌دهد.}

\emph{تام ربات متن قشنگ را ارائه می‌دهد\ldots{}}

\emph{\ldots{} اتفاق خاصی نمی‌افتد. کسی علاقه‌مند به نرم‌افزار زیبای تام
به نظر نمی‌رسد. حتی دوستانش نیز علاقه‌مند نیستند. از آن عده دوستان -همان
۷۰ درصدی که به او گفته بودند که حتما ربات متن قشنگ را استفاده خواهند
کرد- تنها سه نفر پس از یادآوری‌های بسیار آنرا خریداری نمودند. بعد از یک
هفته دونفر آنها نرم‌افزار را از روی گوشی خود پاک کردند و سومین نفر
فراموش کرد که حتی این نرم‌افزار وجود دارد.}

چه اتفاقی افتاده؟

چگونه ممکن است که \textbf{چیز}ی که به این حد نظرات مثبت را به خود جذب
کرده بود به چنین شکستی تبدیل شود. چطور ممکن است چگون 70 درصد تام تبدیل
به 0.002 درصد شد؟ خب این دقیقا همان نتیجه‌ای است که وقتی مبتنی بر آنچه
که شما در سرزمین فکر «یاد گرفته اید» عمل کنید اتفاق می‌افتد.

در این حالت، تحلیل تام که مبتنی بر سرزمین فکر بود از نوع غلط مثبت بود.
تام به زمانی که ایده‌اش در سرزمین فکر بود به این نتیجه رسیده بود که
\textbf{چیز} او یک \textbf{چیز} \emph{درست} است. تام با این فکر که یک
\textbf{چیز} \emph{درست} دارد، کار خود را رها کرده بود و سه ماه را صرف
توسعه یک نرم‌افزار کامل در سه نسخه کرده بود. تام نه تنها مرحله
پیش‌نمونه‌سازی بلکه مرحله نمونه اولیه‌سازی را انجام نداده بود. او از
مرحله ایده مستقیما به مرحله نمونه محصول رفته بود.

نمونه محصول سازی برادر بد طینت پیش‌نمونه سازی است. اگر پیش‌نمونه سازی
بصورت خلاصه «حصول اطمینان از اینکه ساختن \textbf{چیز} \emph{درست} قبل از
ساختن آن \textbf{چیز} بصورت درست» باشد نمونه محصول سازی بصورت خلاصه
«ساختن درست آن \textbf{چیز} حتی اگر مطمئن نیستید که شما آن \textbf{چیز}
\emph{درست} را می‌سازید.»

تام با این کار در چه فکری بود؟ او باهوش بود. چرا او برای ساختن چندین
نسخه بجای یک نسخه سرمایه‌گذاری کرد؟ چرا او نرم‌افزار را چند زبانه طراحی
کرد؟

چیزی که واقعا اتفاق افتاده است این است که بخاطر نظرات مثبت، او قانون
شکست را نادیده گرفته است. او موفقیت را حتمی فرض کرده و تلاش کامل و جامع
را انجام داده است.

این حالت متاسفانه بسیار اتفاق می‌افتد. وقتی شیدایی ما نسبت
\textbf{چیز}مان با نظرات غلط مثبت در سرزمین فکر ترکیب می‌شوند وسوسه
«انجامش دادن» غیر قابل مقاومت است.

از طرف دیگر «انجامش دادن» خیلی خوب به نظر نمی‌رسد؟ آیا شما با گفتن این
مورد و انجام آن حس خوبی ندارید؟ آیا این راه رسم «آمریکایی» نیست؟ بله.
بله. بله. این کار در ابتدا بسیار حس خوبی دارد.

افراد بسیار مثبت نگر تنها افرادی نیستند که به این دام دچار می‌شوند.
کارمندان باتجربه شرکت‌های بزرگ نیز به همان آسانی دچار می‌شوند. آنها از
سرزمین فکر مستقیما به نمونه محصول می‌روند و سقوط می‌کنند.

\textbf{نمونه محصول سازی راهی است که بیشتر محصولات جدید توسط آن ساخته
می‌شوند.}

\textbf{نمونه محصول سازی دلیل اصلی کند و دردناک و پرهزینه بودن شکست‌ها
است.}

\section{«چیز»تان را در اولین فرصت از سرزمین فکر خارج
کنید}\label{ux686ux6ccux632ux62aux627ux646-ux631ux627-ux62fux631-ux627ux648ux644ux6ccux646-ux641ux631ux635ux62a-ux627ux632-ux633ux631ux632ux645ux6ccux646-ux641ux6a9ux631-ux62eux627ux631ux62c-ux6a9ux646ux6ccux62f}

تمام \textbf{چیز}ها چه درست و چه غلط در سرزمین فکر به دنیا می‌آیند. اما
هماهنگونه که دیدیم ماندن زیاد در سرزمین فکر معمولا منجر به رها کردن
ایده‌های خوبمان یا تعهد یا سرمایه گزاری بیش از حد روی ایده‌های بد
می‌شود. به زبان دیگر:

\begin{itemize}

\item
  هیچ کاری در مورد آن \textbf{چیز} انجام ندهید
\item
  آن \textbf{چیز} را انجام دهید(نمونه محصول سازی)
\end{itemize}

همانگونه که می‌دانیم، به احتمال زیاد \textbf{چیز} ما یک \textbf{چیز} غلط
است، اما محلی که از این موضوع اطمینان حاصل کنیم سرزمین فکر نیست بلکه
دنیای واقعی است. جایی که برخلاف نظرات ذهنی، داده‌های واقعی استفاده از
نرم‌افزار و بازار قابل جمع آوری است.

ما نباید بگذاریم که \textbf{چیز}هایمان در سرزمین فکر بگندند. ما بایستی
آنها را از سرزمین فکر در اسرع وقت و با حداقل هزینه خارج کنیم. این
همانجایی است که پیش‌نمونه سازی همان راه سوم و بهترین راه برخورد با دیو
شکست وارد می‌شود:

\begin{itemize}

\item
  آن \textbf{چیز} را امتحان کنید.
\end{itemize}
