پیش نمونه‌های تنها به یک دلیل ساخته می‌شوند و آن دلیل کمک به ما در تعیین
میزان علاقه و عکس العمل مردم به آن \textbf{چیز} ماست. داده‌های که ما به
کمک پیش‌نموه‌ها جمع آوری می‌کنیم به ما کمک می‌کنند که تعیین کنیم ایده‌ی
ما یک \textbf{چیز} \emph{درست} است یا نه.

تنها راه کارا برای دانستن اینکه یک \textbf{چیز} یک \textbf{چیز}
\emph{درست} است تست کردن آن است.این جمع آوری نه در سرزمین افکار با جمع
آوری ایده‌های انتزاعی و نظرات ذهنی صورت می‌گیرد بلکه در دنیای واقعی با
استفاده از یک پیش‌نمونه‌ی ساخته شده از کاربران واقعی صورت می‌گیرد.

\section{داده‌ها بر نظرات
مقدمند}\label{ux62fux627ux62fux647ux647ux627-ux628ux631-ux646ux638ux631ux627ux62a-ux645ux642ux62fux645ux646ux62f}

در گوگل ما چند باور اصلی داریم و آنها «داده‌ها بر نظرات مقدمند»و «آنرا
به کمک اعداد بیان کن» است.

اما ما به کمک پیش‌نمونه‌هایمان چه نوع داده‌ای باید جمع آوری کنیم؟ و
اینکه آنها را به چه اعدادی «بیان» کنیم؟

تقریبا داشتن یک مجموعه ثابت از معیارها که به تمام \textbf{چیز}ها قابل
اعمال باشد غیر ممکن است. به عنوان مثال موفقیت یک کتاب با تعداد فروش آن
اندازه گیری میشود و یک فیلم با فروش گیشه‌ای اش. اما در طرف دیگر موفقیت
یک سرویس تحت وب مثل گوگل یا جیمیل با تعداد افرادی که اسم نویسی می‌کنند
مشخص نشده بلکه از استفاده‌کنندگانی که بصورت متناوب(استفادکنندگانی که ۷
روز هفته را فعالند) از حسابشان استفاده می‌کنند مشخص می‌شود.

در عینی حالی که مجموعه‌ی کلی از معیارهای موفقیت وجود ندارد، خط مشی‌هایی
وجود داشته که به کمک کمی اصلاح به تمامی \textbf{چیز}ها قابل اعمالند.

از آنجایی که خود این کتاب نسخه پیش‌نمونه کتاب است(به بخش مرتبط با محصول
کمینه قابل قبول مراجعه کنید) در اینجا تنها به معرفی دو معیار اولیه اما
مهم می‌پردازم: \emph{سطح علاقه اولیه} و \emph{سطح علاقه مداوم}

\section{سطح علاقه
اولیه}\label{ux633ux637ux62d-ux639ux644ux627ux642ux647-ux627ux648ux644ux6ccux647}

اولین معیاری که شما باید سعی به جمع اطلاعات در موردش برای همه
\textbf{چیز}ها بپردازید چیزی است که من به آن سطح علاقه اولیه می‌گویم.

این معیار یک نسبت بسیار ساده است:

\begin{quote}
سطح علاقه اولیه = تعداد کارهای انجام شده / تعداد کل پیشنهاد انجام آن کار
\end{quote}

که در آن

\begin{quote}
\emph{تعداد کل پیشنهاد انجام آن کار} نماینده تعداد افرادی است که به آنها
پیشنهاد شده است که کاری با پیش‌نمونه‌ی شما انجام دهند و \emph{تعداد
کارهای انجام شده} نشاندهنده تعداد افرادی است که از پیشنهاد شما استقبال
کرده و کاری انجام داده اند.
\end{quote}

مثل همیشه یک مثل به واضح شدن موضوع کمک خواهد کرد.

آدام یک برهنه گراست و سقوط آزاد کننده آماتور است. او در مورد دو «علاقه»
خود بسیار علاقه مند است و در فکر استعفا دادن از کار خود به عنوان یک
حسابدار(جایی که به او اجازه نمی‌دهند برهنه گرا باشد) و خرید یک هواپیما و
شروع اولین کسب و کار سقوط آزاد برهنه در جهان باشد: \emph{سقوط آزاد
مادرزاد}

آدام قبل از اینکه از کار خود استعفا داده و یک هواپیمای ملخی بخرد، بسیار
خوب خواهد بود(اگر بخواهیم تواضع به خرج دهیم) که بدانیم میزان علاقه به
ایده‌ی او چقدر است. آیا سقوط آزاد برهنه یک \textbf{چیز} \emph{درست} است؟
میدانیم که برهنه‌گراها و سقوط آزاد کنندگان بسیاری وجود دارند. آما چقدر
برهنه‌گرا که دوست دارند سقوط آزاد کنند وجود دارد؟ چقدر سقوط آزد کنندگانی
که دوست دارند تنها و تنها با چترشان بپرند وجود دارد؟ کارهایی که آدام
باید برای مشخص کردن میزان علاقه‌ی انجام بدهد از این قرار است.

فروم‌های آنلاین بسیاری برای برهنه‌گراها و سقوط آزاد دوستان وجود دارد.
فرض می‌کنیم که آدام هم‌اکنون عضو چندتا از آنهاست.

آدام ممکن است پستی به این شکل در فروم برهنه‌گراها بفرستد:

\begin{quote}
برهنه‌گراهای عزیز، من میخواهم یک هواپیما برای سقوط آزاد لختی اجاره کنم.
هزینه ۱۰۰ دلار به ازای هر پرش است. نیازی به تجربه قبلی برای سقوط آزاد
نیست و قول میدهم که یک دشت پر خار فروم نخواهیم آمد. اولین پرش یک ماه
بعد(شنبه ۳۱ می) در سانتا باربارا خواهد بود. برای عضوید به من یک ایمیل
فرستاده که حاوی اسامی و تعداد افرادی که در گروه شما هستند باشد. من پاسخ
شما را با جزئیات لازم خواهم داد. ظرفیت محدود است پس اولویت با آنهایی است
که زودتر درخواست داده‌اند.
\end{quote}

\begin{quote}
آدام
\end{quote}
