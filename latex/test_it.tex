پیش نمونه‌های تنها به یک دلیل ساخته می‌شوند و آن دلیل کمک به ما در تعیین
میزان علاقه و عکس العمل مردم به آن \textbf{چیز} ماست. داده‌هایی که ما به
کمک پیش‌نمونه‌ها جمع آوری می‌کنیم به ما کمک می‌کنند که تعیین کنیم ایده‌ی
ما یک \textbf{چیز} \emph{درست} است یا نه.

تنها راه موثر برای دانستن اینکه یک \textbf{چیز}، آن \textbf{چیز}
\emph{درست} است تست کردن آن است.این جمع آوری در سرزمین افکار با جمع آوری
ایده‌های انتزاعی و نظرات ذهنی صورت نمی‌گیرد بلکه در دنیای واقعی با
استفاده از یک پیش‌نمونه‌ی ساخته شده از کاربران واقعی انجام می‌شود.

\section{داده‌ها بر نظرات
مقدمند}\label{ux62fux627ux62fux647ux647ux627-ux628ux631-ux646ux638ux631ux627ux62a-ux645ux642ux62fux645ux646ux62f}

در گوگل ما دو باور اصلی داریم: «داده‌ها بر نظرات مقدمند» و «آنرا به کمک
اعداد بیان کن».

اما ما به کمک پیش‌نمونه‌هایمان چه نوع داده‌ای باید جمع آوری کنیم؟ و
اینکه آنها را به چه صورت عددی «بیان» کنیم؟

داشتن یک مجموعه ثابت از معیارها که به تمام \textbf{چیز}ها قابل اعمال
باشد تقریبا غیر ممکن است. به عنوان مثال موفقیت یک کتاب با تعداد فروش آن
اندازه گیری میشود و یک فیلم با فروش گیشه‌ای اش. اما در طرف دیگر موفقیت
یک سرویس تحت وب مثل گوگل یا جیمیل با تعداد افرادی که اسم نویسی می‌کنند
مشخص نشده بلکه از کاربرانی که بصورت متناوب از حسابشان استفاده می‌کنند
(کسانی که ۷ روز هفته را فعالند) مشخص می‌شود.

در عین حالی که مجموعه‌ای کلی از معیارهای موفقیت وجود ندارد، خط مشی‌هایی
وجود داشته که به کمک کمی اصلاح به تمامی \textbf{چیز}ها قابل اعمالند.

از آنجایی که خود این کتاب نسخه پیش‌نمونه کتاب است(به بخش مرتبط با محصول
کمینه قابل قبول مراجعه کنید) در اینجا تنها به معرفی دو معیار اولیه اما
مهم می‌پردازم: \emph{سطح علاقه اولیه} و \emph{سطح علاقه مداوم}

\section{سطح علاقه
اولیه}\label{ux633ux637ux62d-ux639ux644ux627ux642ux647-ux627ux648ux644ux6ccux647}

اولین معیار را سطح علاقه اولیه می‌نامم. شما بایستی سعی کنید اطلاعات
مرتبط با این معیار را برای همه \textbf{چیز}ها جمع آوری کنید.

این معیار یک نسبت بسیار ساده است:

\begin{quote}
سطح علاقه اولیه = تعداد کارهای انجام شده / تعداد کل پیشنهاد انجام آن کار
\end{quote}

که در آن

\begin{quote}
\emph{تعداد کل پیشنهاد انجام آن کار} نماینده تعداد افرادی است که به آنها
پیشنهاد شده است که کاری با پیش‌نمونه‌ی شما انجام دهند و \emph{تعداد
کارهای انجام شده} نشاندهنده تعداد افرادی است که از پیشنهاد شما استقبال
کرده و کاری انجام داده اند.
\end{quote}

مثل همیشه یک مثال به واضح شدن موضوع کمک خواهد کرد.

آدام یک برهنه گرا و چتر باز آماتور است. او به این «سرگرمی»های خود بسیار
علاقه مند است. تاجایی که او به فکر استعفا دادن از کار خود به عنوان یک
حسابدار(جایی که به او اجازه نمی‌دهند برهنه گرا باشد) افتاده است و
میخواهد یک هواپیما بخرد و اولین کسب و کار چتربازی برهنه در جهان را
راه‌اندازی کند: \emph{چتربازی لخت مادرزاد}

قبل از اینکه آدام از کار خود استعفا داده و یک هواپیمای ملخی بخرد، بسیار
خوب خواهد بود که بداند (اگر بخواهیم تواضع به خرج دهیم) میزان علاقه به
ایده‌ی او چقدر است. آیا جتربازی برهنه یک \textbf{چیز} \emph{درست} است؟
میدانیم که برهنه‌گراها و چتربازهای بسیاری وجود دارند. اما چقدر
برهنه‌گرای علاقه‌مند به جتربازی وجود دارد؟ تعداد چتربازهایی که دوست
دارند تنها لباسشان چترشان باشد، چقدر است؟ کارهایی که آدام باید برای مشخص
کردن میزان علاقه‌ی افراد انجام بدهد از این قرار است:

فروم‌های آنلاین بسیاری برای برهنه‌گراها و چتربازان وجود دارد. فرض
می‌کنیم که آدام هم‌اکنون عضو چندتا از آنهاست.

آدام ممکن است پستی به این شکل در فروم برهنه‌گراها بفرستد:

\begin{quote}
برهنه‌گراهای عزیز، من میخواهم یک هواپیما برای چتربازی لختی اجاره کنم.
قیمت هر پرش ۱۰۰ دلار است. نیازی به تجربه قبلی برای چتربازی نیست و قول
میدهم که یک دشت پر خار فرود نخواهیم آمد. اولین پرش یک ماه بعد(شنبه ۳۱
می) در سانتا باربارا خواهد بود. برای عضویت به من یک ایمیل فرستاده که
حاوی اسامی و تعداد افرادی که در گروه شما هستند باشد. من پاسخ شما را با
جزئیات لازم خواهم داد. ظرفیت محدود است پس اولویت با آنهایی است که زودتر
درخواست داده‌اند.
\end{quote}

\begin{quote}
آدام
\end{quote}

فرض کنیم که آدام یک هفته بعد از ارسال پیغامش متوجه می‌شود که ۱۴۹۰ نفر
پستش را خوانده‌اند(این تعداد کل پیشنهاد‌ انجام کار است) و او فقط ۲ ایمیل
در مورد اینکه آنها می‌خواهند شرکت کنند دریافت کرده است(تعداد کارهای
انجام شده).

مقدار سطح علاقه اولیه در این حالت ۲/۱۴۹۰ = ۰/۰۰۱۳ است. یا ۰.۱۳ درصد.

خیلی دلگرم کننده نیست، البته خیلی هم تعجب برانگیز نیست زیرا اکثر
افراد(شامل برهنه‌گراها) بصورت طبیعی طرفدار پریدن از یک هواپیما سالم
نیستند. در این نقطه آدام می‌تواند به دو پاسخ دهنده بگوید که او متاسف است
و برنامه چتربازی لختی به علت عدم علاقه لغو شده است.

اما آدام قبل از کنارگذاشتن ایده‌اش، یک پست مشابه در فروم چترباز محلی
می‌گذارد. چیزی شبیه این:

\begin{quote}
چتربازان عزیز، آیا شما از راه رسم قدیمی پرش خود خسته نشده‌اید؟ برای
اینکه اوضاع جالب شود من یک هواپیما برای چتربازی لختی اجاره کرده‌ام.
هزینه هر پرش ۱۰۰ دلار است. قول میدهم که در یک دشت پر خار فرود نخواهیم
آمد بلکه در یک ساحل لختی فرود خواهیم آمد، چه هیجان انگیز. اولین پرش یک
ماه بعد(شنبه ۳۱ می) در سانتا باربارا خواهد بود. برای عضویت به من یک
ایمیل فرستاده که حاوی اسامی و تعداد افرادی که در گروه شما هستند باشد. من
پاسخ شما را با جزئیات لازم خواهم داد. ظرفیت محدود است پس اولویت با
آنهایی است که زودتر درخواست داده‌اند.
\end{quote}

\begin{quote}
آدام
\end{quote}

فرض کنیم که بعد از یک هفته ۸۹۸ چترباز پست را خوانده‌اند و ۱۱۲ نفر از
آنها برای پرش اعلام آمادگی کرده‌اند.

میزان علاقه اولیه در این حالت برابر: ۱۱۲/۸۹۸= ۱۲/۵٪ است که عددی بسیار
بزرگتر است.حالا بیایید صحبت کنیم.

بایک پیش‌نمونه درجعلی و به کمک معیار میزان علاقه اولیه در کمتر از یک
ساعت «کار» دوست چترباز برهنه‌گرای ما، آدام، داده‌های با ارزشی جمع آوری
کرده است:

جامعه چتربازان بازار هدف بهتری(۱۰۰ برابر بهتر) نسبت به جامعه برهنگان
است.

میزان علاقه اولیه چتربازان به نسبت بالاست که این عدد بالای ۱۰٪ بوده و با
توجه به جامعه‌ی ۱۰۰۰۰ نفری چترباز در آمریکا به اندازه کافی خوب است که
این ایده را بیشتر مورد بررسی قرار دهیم.

آن درصد از چتربازانی که به آدام ایمیل زده‌اند، بسیار مشتاق بوده و آماده
ثبت نام بودند. این یک سیگنال بسیار قوی در راستای یک \textbf{چیز}
\emph{درست} بودن است.

مقدار علاقه اولیه بسیار قوی و به راحتی قابل تفسیر و مقایسه در برابر یک
مقدار علاقه اولیه دیگر است. در مورد آدام، میزان علاقه اولیه بصورت غیر
مبهمی بیان میدارد که چتربازان بازار هدف بهتری از برهنه‌گراها هستند. اما
دانستن آنکه این سطح علاقه اولیه به اندازه کافی خوب است که ادامه داد یا
نه کار سختی است. برای برخی از \textbf{چیز}ها میزان علاقه اولیه ۱۲/۵
درصدی ممکن است عالی در نظر گرفته شود اما برای برخی دیگر اینگونه نیست.
باید توجه داشت در عین حالی که جمع آوری اطلاعات برای محاسبه میزان علاقه
اولیه مهم است، تفسیر آن نیاز به قضاوت و دانش آن حوزه یا بازار دارد.

اوضاع ایده \emph{چتربازی لخت مادرزاد} خوب به نظر می‌رسد اما میزان علاقه
اولیه یک نشانگر اولیه برای یک \textbf{چیز} \emph{درست} بالقوه است.
بیایید چیزی که آدام باید آنرا پیش‌نمونه‌سازی کند و اندازه بگیرد را بررسی
کنیم.

\begin{quote}
\textbf{توجه:} من یک حس غریب در مورد اینکه چتربازی برهنه برخلاف قوانین
هوایی باشد دارم. از آنجایی این کتاب نیز خود یک پیش‌نمونه است، من
تحقیقاتی جامعی در این مورد انجام نداده‌ام. و برای اینکه مطمئن باشم
می‌گویم که من به هیچ‌وجه ایده‌ی چتربازی برهنه را پیشنهاد نکرده و بر آن
صحه نمی‌گذارم پس آنرا در خانه امتحان نکنید. اما اگر کردید، من را برای
انجام این ایده شماتت نکرده و عکس خود را برای من نفرستید.
\end{quote}

\section{سطح علاقه
مداوم}\label{ux633ux637ux62d-ux639ux644ux627ux642ux647-ux645ux62fux627ux648ux645}

برای برخی از \textbf{چیز}ها، موفقیت وابسته به تکرار کسب و کار است(مثلا
یک کتاب یا یک بازی آرکید). میزان خوب سطح علاقه اولیه ممکن است کافی باشد
تا کار را پیش بگیریم. اما \textbf{چیز}های بسیاری هستند که موفقیتشان
وابسته به تکرار خرید، بازدید مجدد، یا استفاده مداوم توسط گروهی از افرادی
است که بصورت اولیه علاقه‌مند به آن \textbf{چیز} بوده‌اند. مخصوصا هنگامی
که راه‌اندازی کسب و کار نیاز به پیش خرید تجهیزات گران‌قیمت یا هزینه
سنگین تکرار شونده دارد.

برخلاف میزان علاقه اولیه، سطح علاقه مداوم بجای یک عدد توسط نمودار(جدول)
مبتنی بر زمان به نمایش گذاشته می‌شود. هر مدخل یا نقطه در این نمودار یا
جدول میزان علاقه در یک تاریخ خاص است. ما بهتر است دنبال چه معیاری در
جدول یا گراف سطح علاقه مداوم باشیم؟ آیا علاقه در طول زمان به صفر میل
کرده است؟ آیا یک مقدار کاهش یافته سپس در سطح قابل قبولی به ثبات می‌رسد؟
آیا افزایش می‌یابد؟ در اولین حالت شما احتمالا با یک \textbf{چیز}
\emph{غلط} سروکار دارید. در حالت دوم ممکن است اوضاع بهتر یا بدتر شود و
نیاز به بررسی بیشتری دارد. حالت سوم یک نشانه امیدوار کننده است و ممکن
شما یک \textbf{چیز} \emph{درست} داشته باشید.

مثل همیشه توضیح دادن با استفاده از یک مثال بسیار ساده‌تر است. بیایید از
جایی که مثال قبل با آدام را رها کرده‌ بودیم را از سر بگیریم و به سراغ
کسب و کار چتربازی برهنه را از سر بگیریم.

در مورد \emph{چتربازی لخت مادرزاد} آدام آدم بی توجه خواهد بود اگر تنها
بر اساس معیار سطح علاقه اولیه از شغلش استعفا داده و یک هواپیمای ملخی
بخرد. حتی اگر بیش از ۱۰\% تمام چتربازان علاقه‌مند به امتحان یک پرش برهنه
هستند اما اگر آنها برای انجام دوباره اینکار برنگردند این یک کسب و کار
کوتاه خواهد بود.

آدام قبل از گرفتن هر تصمیم(مثل استعفا از شغلش) یا سرمایه(مثل خرید یک
هواپیما ملخی) بزرگ، بایستی سطح علاقه مداوم را اندازه گیری کند.

پیش‌نمونه در جعلی برای تست علاقه اولیه بسیار خوب است اما برای تست سطح
علاقه مداوم به چیزی ملموس‌تر و قابل‌توجه‌تری نیاز است. بیشتر افراد به
باز کردن در جعلی ادامه نخواهند داد. پیش‌نمونه وانمود کردن دارایی در این
مورد کارا خواهد بود.

بجای خرید یک هواپیما، آدام بایستی هواپیما را در موارد مورد نیاز اجاره
کند. اجاره کردن روزانه هواپیما به عنوان یک انتخاب دراز مدت برای
\emph{چتربازی لخت مادرزاد} پرهزینه و غیر عملی است. اما تا زمانی که آدام
متقاعد شود ایده چتربازی برهنه‌ی او موفق خواهد بود خرج کردن چند صد دلار
اضافه برای تست آن بجای خرج ده‌ها هزار دلار سرمایه‌گذاری به امید داشتن یک
\textbf{چیز} \emph{درست} بهتر است. \emph{قانون شکست} را باوجود میزان
علاقه اولیه مثبت به خاطر بیاورید که شانس بر علیه \textbf{چیز} آدام است.

بیایید فرض کنیم که آدام از پروتکل پیش‌نمونه‌سازی پیروی کرده و تبلیغات
خود را در فروم چتربازان هر هفته ادامه داده و در طول دو ماه ۸ پرواز انجام
میدهد: یک پرواز در هر شنبه

دادهای مرتبط با سطح علاقه مداوم برای دو ماه به شرح زیر است

\begin{longtable}[c]{@{}lcrll@{}}
\toprule\addlinespace
شماره پرواز & تعداد ثبت نام & درآمد & هزینه & سود/زیان
\\\addlinespace
\midrule\endhead
۱ & ۲۱ & ۲۱۰دلار & ۲۵۰دلار & -۴۰ دلار
\\\addlinespace
۲ & ۲۰ & ۲۵۰دلار & ۲۵۰دلار & ۰ دلار
\\\addlinespace
۳ & ۲۸ & ۲۸۰دلار & ۲۵۰دلار & ۳۰ دلار
\\\addlinespace
۴ & ۱۷ & ۱۷۰دلار & ۲۵۰دلار & -۸۰ دلار
\\\addlinespace
۵ & ۷ & ۷۰دلار & ۲۵۰دلار & -۱۸۰ دلار
\\\addlinespace
۶ & ۳ & ۳۰دلار & ۲۵۰دلار & -۲۲۰ دلار
\\\addlinespace
۷ & ۰ & ۰دلار & ۰دلار & ۰ دلار
\\\addlinespace
۸ & ۰ & ۰دلار & ۰دلار & ۰ دلار
\\\addlinespace
جمع & ۱۰۱ & ۱۰۱۰دلار & ۱۵۰۰دلار & -۴۹۰ دلار
\\\addlinespace
\bottomrule
\end{longtable}

آدام متاسفم! اوضاع یک مدت خوب به نظر میرسید - حتی توانستی در سومین پرواز
به اندکی سود برسی - اما میترسم که چتربازی برهنه یک \textbf{چیز}
\emph{درست} نباشد.

مقدار بالای سطح علاقه اولیه بسیار خوب است اما اگر موفقیت \textbf{چیز}
شما به کارکرد مدوام احتیاج دارد، در صورتی \textbf{چیز} شما نیاز به
سرمایه گذاری قابل توجهی دارد، شما بایستی سطح علاقه مداوم را نیز باید تست
کنید. در مورد آدام، پیش‌نمونه‌سازی پیشنهاد می‌دهد که \emph{چتربازی لخت
مادرزاد} به عنوان یک علاقه جانبی و لذت بخش قابل قبول است، اما در حالت
کنونی استعفا از کار، خرید یک هواپیما و سعی در گذران زندگی با استفاده از
آن کار غیر معقولی به نظر میرسد. پیش‌نمونه‌سازی او را نجات داد و همچنین
ما را از خطر حضور چتربازان لخت را در حیاطمان حفظ کرد.
