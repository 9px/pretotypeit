این یک کتاب به شکل معمول نیست.

نوشتن و ویرایش یک کتاب به شکل معمول در مورد پیش‌نمونه سازی ماه‌ها زمان
خواهد برد. من دوست دارم اینچنین کتابی به نویسم اما در حال حاضر نشانه‌ای
بر ارزشمند بودن نوشتن چنین کتابی وجود ندارد. بیشتر کتاب‌ها در بازار شکست
می‌خورند، و دلیل شکست اکثر آنها این نیست که به درستی نوشته نشده یا
ویرایش نشده‌اند، بلکه به این دلیل است که افراد کمی به آنها علاقه‌مند
هستند. آنها \emph{یک} \textbf{چیز} \emph{درست} نیستند.

کتابی که پیش روی شماست نسخه پیش‌نمونه‌ی کتاب است. من این کتاب را در عرض
چند روز نوشتم و «ویرایش» کردم بجای چند ماه، به منظور اینکه سطح علاقه به
این کتاب را دریابم. برخی از دوستان و همکاران من این کتاب را بررسی
کرده‌اند اما اگر در این کتاب غلط املایی، دستور زبان نادرست و هرگونه
\emph{ایراد} دیگر پیدا کردید تعجب نکنید.

نشر این کتاب در این وضعیت برای من آسان نیست.

سخت ترین بخش در مورد پیش‌نمونه سازی توسعه پیش‌نمونه ها نیست زیرا این بخش
لذتبخش است. سخت ترین بخش غلبه بر میل شدید به ایده‌آل گرایی و همچنین
علاقه به اضافه کردن ویژگی و یا محتوا قبل از انتشار اولیه است. بخش سخت
عرضه پیش‌نمونه در مقابل دیگران است و این در حالی است که ممکن است مورد آن
قضاوت شود، مورد نقد قرار بگیرد و یا بصورت محتمل طرد گردد.

رید هافمن -یکی از پایه گذاران لینکدین- می‌گوید: «اگر شما از اولین نسخه
محصول خود خجالت نمی‌کشید و به آن افتخار می‌کنید شما خیلی دیر نسخه اولیه
را ارائه کرده‌اید»

من خیلی خجالت میکشم، پس من باید مسیر درستی را انتخاب کرده باشم.
